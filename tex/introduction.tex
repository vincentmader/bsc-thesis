\chapter{Introduction}

  \pagenumbering{arabic} 
  \setcounter{page}{1} % TODO: make sure this is set correctly

  \section{Historical Context}

%    \red{For a long time, planets have been objects of interests for mankind.} \\
%    \\
    As early as the time of the ancient Babylonian civilization, several 
    bright objects could be distinguished from the rest of the the night sky. 
    In contrast to the stars, which on human timescales constitute a relatively
    static background, these objects seemed to exhibit independent motion 
    across 
    the sky. This explains why the word we use today to describe these objects 
    is \textit{planet}, which has its origin in the ancient Greek 
    \textit{planetes}, literally meaning "wanderer".
    It was observed that every now and then, the movement of these sources of 
    light seemed to stop and then reverse for a while. Nowadays, we of course 
    know that this behavior is explained by the planets' 
    independent motion around the Sun. Although it was suggested that the Earth 
    does not make up the center of the universe relatively early, e.g. by
%    the Pythagoreans in the third century B.C.E. as well as by 
    Aristarchus of Samos in the third century B.C.E., 
    this was not a widely accepted fact until relatively recent times. \\
    \\
%
%    Although a heliocentric model of the 
%    universe was suggested 
%
%    Pythagoreans 
%
    Up until the Renaissance, the widespread world view followed a model most 
    notably suggested by the Greek polymath Ptolemy, who argued for a geocentric 
    universe, where the Sun, planets and stars all orbit the Earth. Each of 
    the orbiting planets was thought to additionally travel on a so-called 
    epicycle, which helped explain the fact that the observed planets did not 
    always travel in the same direction in the sky as seen from Earth. \\
    \\
    Due to their low apparent magnitudes, the planets Mercury, Venus, Mars, 
    Jupiter and Saturn can be observed from Earth with the naked eye. Thus, 
    they were known to most historic societies with any interest in astronomy.
    Not being able to explain their origin, humans in civilizations all over 
    the world tried to give these mysterious sources of light meaning by 
    incorporating them into their stories, mythologies and religions. \\
    \\
%    The six inner planets were known to man since ancient times (even though 
%    Earth was not regarded as residing in the same category as the others).
%    Their low apparent magnitudes makes them visible from Earth without 
%    any help from telescopes. \\
%    \\
%    The planets' true nature could not be explained\red{...}, so people 
%    tried to give them meaning in their stories. Thus, they played a role in the
%    religion and mythology of many different civilizations across the globe.
%    \\
    It would take almost two millenia before the heliocentric model of the solar 
    system gained widespread attention. In the 16th century, the Polish 
    polymath Nicolaus Copernicus formulated a model of the universe
    with the Sun at its center and the planets traveling around it on 
    concentric circles. 
    Eventually, the observations of Mars' orbit done 
    by Johannes Kepler and his mentor Tycho Brahe lead to a new empirical
    description of the motions of astronomical bodies that improved on the 
    model of Copernicus, known to us today as Kepler's laws. Among other 
    things, Kepler realized that the planets move around the sun not on circles,
    but on ellipses, with the Sun positioned at one of the foci. \\
    \\
    These observations would lead Isaac Newton to formulate his theory of 
    gravity together with his laws of motion
    which, it could be argued, marked the start of the scientific 
    revolution that would take place in the following centuries. 
%    After 
%    stating his three laws of motion, he formulated a law of gravity, which 
%    which can be expressed as 
%    \begin{equation}
%      \vec F_G=G\cdot\frac{m_1\cdot m_2}{|\vec r_1-\vec r_2|^2}\cdot 
%      \frac{\vec r_1-\vec r_2}{|\vec r_1-\vec r_2|}
%    \end{equation}
%    It quantifies the gravitational force $F_G$ acting between any two bodies 
%    with masses $m_1$ and $m_2$ and position vectors $\vec r_1$ and $\vec r_2$.
%    The proportionality constant $G$ is also known as Newton's gravitational 
%    constant. 
    He recognized that a simple inverse square law suffices to predict 
    the motions of the astronomical bodies in a relatively accurate way.
    Additionally, he noticed that this
    same law held true not only for the planets, but also for any falling 
    object here on Earth. 
    % This can be regarded as the discovery of the first scientific law. 

    \newpage \noindent
    The new heliocentric world view provided a lot of new questions in need of 
    answers during the following centuries. Are all stars actually suns like 
    our own, only much farther away? Could there be planets orbiting those 
    stars and could those planets look like our own home planet, possibly 
    even with their own unique life forms and evolutionary histories? \\
    \\
    An attempt to explain the origin of the Solar System was made in the 18th
    century in part by Immanuel Kant and Pierre-Simon Laplace in their
    % TODO: how much citing in historical context? ask Bertram
    \textit{nebular hypothesis}, in which they
    argued for the former existence of a giant gas cloud. Slowly rotating,
    this cloud was suspected to have collapsed and subsequently flattened out
    due to its own gravity. Later on, the Sun and the planets were thought 
    to have emerged from the gas. \\
    \\
    After the invention of the first (semi-)modern telescopes by 
    Galileo and Newton, the following centuries saw the discoveries of more 
    planets in our own solar system.
    First Uranus in the late 1700s, then Neptune a century later, bringing the 
    number of known planets in our solary system to eight.
    The discovery of Pluto and later many other objects of the same size in the 
    outer solar system prompted the \textit{International Astronomical Union} to 
    formally define the term \textit{planet} in 2006 \cite{IAU_planet_def}.
    For any object in the Solar System to be classified as a planet, it has to
    \begin{enumerate}
      \setlength{\itemsep}{2pt}%
      \setlength{\parskip}{0pt}%
      \item be in direct orbit around the Sun
      \item possess a mass large enough to assume a nearly round shape due 
        to self-gravitation
      \item have cleared its orbit of any larger or similarly-sized bodies
    \end{enumerate}
    With continuing scientific and technological progress, telescope resolution 
    improved drastically, increasing the distance at which astronomical 
    bodies could be observed. Larger and more 
    accurate telescopes, as well as the eventual possibility of launching 
    these telescopes into orbit made the discovery of extrasolar planets 
    feasible. The first discovery of such an exoplanet was confirmed by
    \citeauthor{Wolszczan_1992} (\citeyear{Wolszczan_1992}) 
    % \cite{first_exoplanet_discovered} 
    with about 5000 having been observed since then, most 
    notably by the \textit{Kepler Space Telescope}, with which alone about 2600
    planets could be detected. The first direct observation of a 
    nascent proto-planet inside a proto-planetary disk was reported in 2018,
    after the exoplanet \textit{PDS 70b} was imaged using ESO's 
    \textit{Very Large Telescope} (\citeauthor{Keppler_2018},
    \citeyear{Keppler_2018}).
    \\
    \\
    These recent observations show that planets are very abundant. 
    There might be more stars with an own 
    planetary system than without, as has been stated in a 2018 article by 
    NASA \cite{more_stars_with_planets_than_without}.
    Observations by e.g. \citeauthor{Petigura_2013} 
    (\citeyear{Petigura_2013}) indicate that a significant portion of 
    stars is accompanied by Earth-sized planets 
    % with orbital periods similar 
    % to that of the Earth.
    inside the Goldilocks zone, where liquid water can exist in a stable 
    form and therefore one of the basic requirements for life is fulfilled.
    Extrapolating from these observations, it is possible that there could up 
    to 40 billion Earth-like planets in the Milky Way alone. \\
    \\
    The mass of observed planets ranges from about twice that of the Moon 
    to about 30 times that of Jupiter
    \footnote{Deuterium burning starts at about 13 Jupiter masses, making these 
    larger objects brown dwarfs.}, thus spanning multiple orders of 
    magnitude. 
    Planetary compositions depend largely on their size.
    Most modern planetary formation theories assume an initial buildup of a 
    central rocky core. From there the planet evolves either into an Earth-like 
    terrestrial planet or into a gaseous giant similar to Jupiter
    (\citeauthor{Pollack_1996}, \citeyear{Pollack_1996} and 
    \citeauthor{Kley_1999}, \citeyear{Kley_1999}). \\
    \\
    % \red{Rogue planets: might be more rogue planets than stars in the Milky Way 
    % \cite{cite}.}
    % \red{...} \\
    % \\
    The processes underlying planet formation are yet to be fully explained. 
    Accurate modeling requires an understanding of many different 
    aspects of the natural sciences, including but not limited to 
    magneto-hydro-dynamics, chemistry, gravity (N-body-dynamics), 
    thermodynamics, radiative transfer and coagulation physics. Due to temporal 
    constraints and the complexity of the topic, this bachelor's thesis can 
    only cover a small portion of what there is to be said about this 
    % interesting and 
    active field of study.
%    \\ \\
%    \red{The following sections are an attempt to convey the essentials of 
%    the physical model which is used in this thesis to model a proto-planetary
%    disk}
%    \begin{itemize}
%      \item formation of solar systems/planets not entirely understood yet
%        $\Rightarrow$ active field of study, complex topic 
%        (gravity, magneto-hydro-dynamics, material science (?), ...)
%      \item temporal limits, this bachelor's thesis can only cover a tiny portion 
%        of what there is to be said about this complex and actively evolving field 
%        of study
%      \item this work utilizes construct a highly simplified model of a 
%        proto-planetary disk (2D) with the help of computer simulations 
%        (FARGO2D1D). it is assumed that a ... planet has already formed 
%        (circumventing problem of early accretion, cite) that is orbiting a central 
%        proto star
%      \item the evolution of the planet is obviously greatly 
%        influenced/characterized by the rate of mass accretion. 
%      \item variation of disk and planet parameters $\Rightarrow$
%        influence on evolution of planet
%    \end{itemize}
  
  %\section{\red{Prerequisites}}
  %  \subsection{Classical Gravity}
  %    
  %    \paragraph{Kepler's Laws}

  %      \begin{itemize}
  %        \item formulated by Johannes Kepler between 1609 and 1619 
  %        \item empirical laws from studies of Tycho Brahe's observational data
  %        \item first two laws were published in 1609, \textit{Astronomia Nova}
  %          \begin{enumerate}
  %            \item A planet travels around its parent star on an ellipse.
  %            \item The planet sweeps out equal areas in equals times 
  %          \end{enumerate}
  %        \item additionally, Kepler regognized that neither velocity nor 
  %          angular velocity are constant, but area velocity is 
  %          (closely related to angular momentum)
  %        \item ellipses are conical sections, characterized by the equation
  %          \begin{equation}
  %            \frac{x^2}{a^2}+\frac{y^2}{b^2}=1,\ \ \ \ \ \textnormal{with }a>b
  %          \end{equation}
  %        \item ellipse characterized by two numbers 
  %          ($a$ \& $b$ or one axis and $e$)
  %        \item Earth's orbit has an eccentricity of about $0.017$
  %          \begin{equation}
  %            e=\sqrt{1-\frac{b^2}{a^2}}
  %          \end{equation}
  %        \item \red{give equation for trajectory}
  %        \item circles can be seen as special cases of ellipses, or as a
  %          separate class of conical section
  %        \item can also be parabolas or hyperbolas ($e=1$ or $e\geq1$)
  %        \item third law published in 1619
  %        \item square of the orbital period of a planet is directly 
  %          proportional to the cube of the semi-major axis of its orbit.
  %      \end{itemize}

  %    \paragraph{Newton's Law of Gravitation}
  %      \begin{itemize}
  %        \item published in Isaac Newton's famous $Principia$
  %        \begin{equation}
  %          \vec{F}=G\cdot\frac{m_1\cdot m_2}{|\vec{r}_2-\vec{r}_1|^2}\cdot
  %          \frac{\vec{r}_2-\vec{r}_1}{|\vec{r}_2-\vec{r}_1|}
  %        \end{equation}
  %        \item from this equation, Kepler's empirical laws could be 
  %          derived mathematically/theoretically (theoretical underpinning)
  %        \item inverse square law is responsible for first and third 
  %          Kepler law, conservation of angular momentum for second one
  %        \item Vis Viva equation \cite{vis_viva_equation}
  %        \item gravitational potential created by central star and planet is
  %          \begin{equation}
  %            \Phi=\frac{GM_S}{|\vec r-\vec r_S|}+\frac{Gm_P}{|\vec r-\vec r_P|}
  %          \end{equation}
  %        \item for near circular orbit, centrifugal and gravitational forces 
  %          are equal ($m$ is mass of particle, gas? planet? what 
  %          about $e\neq0$?)
  %          \begin{equation}
  %          \frac{mv^2}{r}=G\cdot\frac{M_S\cdot m_p}{r^2}
  %          \end{equation}
  %        \item solving for $v$ gives so-called Kepler velocity 
  %          \begin{equation}
  %            v_{Kepler}=\sqrt{\frac{G\cdot M_S}{r}}
  %          \end{equation}
  %        \item rewrite with $\Omega_{Kepler}$
  %        \item this is equivalent to Kepler's third law, can be shown easily 
  %          with $\Omega=\frac{2\pi}{T}$
  %      \end{itemize}
  %  
  %      \myparagraph{The Roche Limit \red{(relevant? Roche Lobe vs. Roche Limit?)}}
  %        Let us consider two identical masses $m$ in orbit around a central mass
  %        $M$. The central body exerts a gravitational force on both of the
  %        orbiting masses, which can be expressed as
  %        \begin{equation}
  %          F_i=G\cdot\frac{M\cdot m}{r_i^2}
  %        \end{equation}
  %        Here, \red{$G$ labels the Newtonian gravitational constant and} $r_i$ is 
  %        the distance from the central body to one of the two orbiting masses,
  %        characterized by the index $i\in\{1,2\}$. The \textit{tidal force}
  %        \begin{equation}
  %          \Delta F=F_2-F_1
  %        \end{equation}
  %        is defined as the difference of these two forces. Whether or not the two
  %        bodies form a bound system relies on whether or not the tidal force is 
  %        greater than the mutual gravitational attraction between the two bodies, 
  %        which can be written as 
  %        \begin{equation}
  %          F_{12}=G\cdot \frac{m^2}{r_{21}^2},
  %        \end{equation}
  %        where $r_{12}$ denotes the distance between the two orbiting masses. The 
  %        Roche limit is reached when the following relation holds:
  %        \begin{equation}
  %          F_{12}=\Delta F,
  %        \end{equation}
  %        which can be expanded into
  %        \begin{equation}
  %          G\cdot M\cdot m\cdot(\frac{1}{r_2^2}-\frac{1}{r_1^2})
  %          =G\cdot \frac{m^2}{r_{12}^2}
  %        \end{equation}
  %        and then simplified to 
  %        \begin{equation}
  %          M\cdot(\frac{1}{r_2^2}-\frac{1}{r_1^2})
  %          =m\cdot\frac{1}{r_{12}^2}
  %        \end{equation}
  %        \begin{equation}
  %          r_{12}^2\frac{r_1^2-r_2^2}{r_1^2\cdot r_2^2}
  %          =\frac{m}{M}
  %        \end{equation}
  %        \red{...} \\
  %        \red{masses will not be bound if $r_{12}\leq r_{crit.}$} \\
  %        \red{(later used when FARGO2D1D accretes material from inside Roche lobe?)}
  %  

  \newpage
  \section{Proto-Planetary Disks}
%    Observations of young stars have verified the nebular hypothesis
%    Since the early 1980s, studies of young stars have shown them to actually 
%    be surrounded by disks of dust and gas, just as the nebular hypothesis 
%    suggested. 
%    The first image of a proto-planetary disk was reported in
%    July 2018 \red{(PDS 70b)} \red{[cite]}. 
%    Since then, \red{how many} 
%    proto-planetary disks have been observed \red{[cite]}.
%    \begin{figure}[h!]
%      \centering
%      \includegraphics[width=.5\textwidth]{ALMA_eso1611a}
%      \caption{
%        proto-planetary disk around the star TW Hydrae, observed with ALMA
%        \cite{proto-planetary_disk_around_TW_Hydrae_image}
%        \red{(more images)}
%      }
%      \label{fig:proto-planetary_disk_around_TW_Hydrae}
%    \end{figure} \ \\ 
%    From observations like this one many insights into the evolution of 
%    extra-solar planets as well as the origin of planets in our own Solar 
%    System can be acquired. \\
%    \\
    The current scientific consensus on the origin theory of stars and planets 
    is based on the nebular hypothesis, which states that proto-planetary disks 
    form when large interstellar clouds of molecular gas collapse under 
    their own gravity (\citeauthor{Woolfson_1993}, \citeyear{Woolfson_1993}).
    The chemical composition of these disks is dominated by the contribution of 
    \ch{H2}, which in most cases makes up about 98\% of the disk by 
    mass. Besides that, there are also small quantities of helium, lithium and 
    trace amounts of heavier elements. In observed disks, most of the matter is 
    present in the form of gas, but some of it also exists as dust.
    The exact ratio of dust to gas and the distribution of these two components 
    throughout the disk is still largely unknown (\citeauthor{Birnstiel_2010}, 
    \citeyear{Birnstiel_2010} and \citeauthor{Soon_2019}, \citeyear{Soon_2019}).
%    In the early universe, these clouds formed from the material created in 
%    the Big Bang after the expansion 
%    Shortly after the Big Bang, most of the matter in the universe consisted of 
%    hydrogen along 
%    \red{Origin of clouds?} \\
    \\
    \\
    The collapse of such a cloud leads to a drastic increase of gas density, 
    pressure and temperature at its center, which for high mass clouds results 
    in the formation of a star. 
    Conforming to the law of momentum 
    conservation, the cloud's angular velocity increases during the collapse.
    Also, the initially present statistical distribution of velocities averages 
    out in favor of the clouds's net angular momentum.
    The centrifugal force present in a rotating reference frame can only 
    balance out the pull of gravity along the radial axis. Therefore, the cloud
    flattens out into a disk much more wide than thick, supported
    only by gas pressure along the axis orthogonal to the disk. 
    Typical disk radii are on the order of a few 100 AU 
    (\citeauthor{Pfalzner_2015}, \citeyear{Pfalzner_2015}).
    % In our own Solar System, this collapse and flattening process took 
    % approximately \red{x years} 
    % \cite{cite}. 
    \\
    \\
    The flattened cloud can also be regarded as an accretion disk around the 
    central star, which after an initial accretion phase 
    % on timescales of approximately \red{x years} 
    makes up almost all ($>$ \SI{99}{\percent}) of the mass in the disk. 
    % \cite{cite}
    Planets form from the remaining gas and dust in orbit around the star. 
    % This gas heats up due to friction and radiates 
    % away energy, which can be observed with submillimeter telescopes.
    \autoref{fig:proto-planetary_disk} shows two images of proto-planetary disks 
    that were observed at the 
    \textit{Atacama Large Millimeter/submillimeter Array}
    (ALMA). The images show the thermal emission of millimeter-sized grains.
    Multiple circular dark regions can be seen, which could be caused by 
    planets that push the dust particles aside due to the influence of the 
    planet's gravity on the disk structure.
    (e.g. \citeauthor{Paardekooper_2006}, \citeyear{Paardekooper_2006}).
    % Those are
    % regions where the gas has partially been depleted by an exchange of
    % angular momentum with the planet.
%    % TODO: does Fig. 1.1.b render correctly?
%    Figure\autoref{fig:proto-planetary_disk_around_TW_Hydrae:b} 
%    shows the
%    proto-planetary disk around the start \textit{TW Hydrae}. At a distance of 
%    roughly 175 light years, it is relatively close to the Earth. 
%    Additionally, its quite young age of 
%    about 10 million years and relative inclination of roughly 90 degrees 
%    \red{(face-on view)} make it a good target for observation. \\
%    \\
%    In our own Solar System, the Sun 
%    makes up $<$ \SI{99}{\percent} of the total mass.
%    \begin{itemize}
%      \item collapse takes about 100.000 years
%        $\Rightarrow$ then, star has similar temperature to main sequence 
%        star of same mass, becomes visible
%      \item oldest disk ever observed: 25 million years
%        % White, R.J. & Hillenbrand, L.A. (2005). "A Long-lived Accretion Disk
    %        around a Lithium-depleted Binary T Tauri Star". The Astrophysical
    %        Journal. 621 (1): L65--L68. arXiv:astro-ph/0501307. Bibcode:2005ApJ...621L..65W. doi:10.1086/428752. - Cain, Fraser; Hartmann, Lee (3 August 2005). "Planetary Disk That Refuses to Grow Up (Interview with Lee Hartmann about the discovery)". Universe Today. Retrieved 1 June 2013.
%      \item after that, gas is either blown away by stellar wind or simply stops emitting radiation
%    \end{itemize} \ \\
%    The nebular hypothesis forms the basis for the current scientific consensus
%    \red{to explain} the origin of stars and planets. It states that
%    proto-planetary disks come into existence when large interstellar clouds of 
%    molecular gas collapse under their own gravity. \red{origin of the clouds?}
    \\
    \\
    \begin{figure}[h!]
      \centering
      \begin{minipage}{.5\linewidth}
        \centering
        \subfloat[Disk around \textit{HL Tau} \cite{ALMA_HL_Tau}]{
          \label{fig:proto-planetary_disk:a}
          \includegraphics[width=.9\linewidth]{ALMA_HL_Tau}
        }
      \end{minipage}%
      \begin{minipage}{.5\linewidth}
        \centering
        \subfloat[Disk around \textit{TW Hydrae} \cite{ALMA_TW_Hydrae}]{ 
          \label{fig:proto-planetary_disk:b}
          \includegraphics[width=.9\linewidth]{ALMA_TW_Hydrae}
        }
      \end{minipage}
      \caption{Proto-planetary disks around the stars \textit{HL Tau} and 
        \textit{TW Hydrae}, observed at submillimeter wavelengths at
        \textit{ALMA}. The dark rings indicate regions of lower gas density,
        hinting at the existence of exoplanets. 
      }

      \label{fig:proto-planetary_disk}
    \end{figure}
    %The direction of the net 
    %angular momentum vector defines the disk's orientation, since the gas is 
    %free to fall towards the center along any \red{position component} not 
    %parallel to the radial vector \red{gravity vs. centripetal force}.
    %\red{This leads to a thin disk only supported by gas pressure along the 
    %$z$-axis}. 

    \newpage \noindent
    To construct a simplified model of a proto-planetary disk in two dimensions,
    we use cylindrical coordinates $(r,\varphi,z)$. In this model, the 
    disk is situated at $z=0$ and is infinitesimally thin. Since observations 
    show proto-planetary disks to be much more wide than thick anyways, this 
    assumption makes it possible to obtain a relatively accurate, yet much 
    less computationally constraining model. It allows us to ignore the 
    $z$-dependence of the parameters describing the disk, including gas 
    density, temperature and rotational velocity. Going from a 3D model to a 
    2D model in this way naturally suggests the definition of a surface density
    \begin{equation}
      \Sigma(r,t)=\int_{-\infty}^{\infty}\rho(z,t)\cdot dz
    \end{equation}
%    Because the disk is much more wide then thick, it is possible to obtain 
%    a relatively accurate, yet less computationally \red{aufwaendig} model by 
%    vertically integrating the hydrodynamical equations and only working
%    with vertically \red{averaged} state variables \cite{cite}. 
%    This greatly simplifies \red{what?}.
%    \\
%    \\
    In this highly simplified model, all inhomogenities in the chemical 
    composition of the disk are neglected. The disk is assumed to consist 
    entirely of individual gas particles with a mean molecular mass $m_{mol}$.
    The temperature is vertically isothermal and under the assumption of a 
    constant aspect ratio (which is defined below) is proportional to $1/r$.
    % and uniform temperature $T$.
    If the gas were perfectly Keplerian, the trajectory of such a gas 
    particle in the gravitational potential of a star with mass $M_*$ would
    be described by Kepler's laws. Assuming a circular orbit, the angular 
    frequency could then be expressed as
    \begin{equation}
      \Omega_K=\sqrt{\frac{G M_*}{r^3}}
      \label{eq:kepler_freq}
    \end{equation}
    In reality, the gas in a disk does not orbit its parent star in a 
    perfectly Keplerian way, 
    % since the pressure gradient 
    % non-neglectable viscosity of the gas
    % leads to slightly 
    with lower rotation velocities than one would expect from 
    \autoref{eq:kepler_freq}. This is due to the pressure gradient in the disk, 
    which effectively leads to a decreased gravity felt by the gas.
    If, additionally, a planet is present in the disk 
    and its interactions with the gas are taken into account, the trajectories 
    of individual bodies (planet or gas particles) in this system become 
    impossible to determine analytically, therefore making the utilization
    of numerical simulations very attractive.
    \\ %\red{moving coordinate system}
    \\
%    much more wide then thick (to be expected for near-Keplerian rotation)
%    $\Rightarrow$ vertically integrate hydrodynamical 
%    equations, work only with vertically averaged state variables \\
%    \begin{figure}[h!]
%      \centering
%      \includegraphics[width=\textwidth]{proto-planetary_disk_early_evolution_sketch}
%      \caption{\red{early evolution of proto-planetary disk}}
%    \end{figure} \ \\ 
    Since they mostly probe the outer domains of proto-planetary 
    disks, millimeter observations as of yet do not give much information about 
    the gas density profile in the inner regions of the disk
    (\citeauthor{Dullemond_2010}, \citeyear{Dullemond_2010}). 
    A first approach to modeling the gas
    surface density's dependency on $r$ is to assume a simple power law:
    \begin{equation}
      \Sigma(r)=\Sigma_0\cdot\bigg(\frac{r}{r_0}\bigg)^{-\gamma}
      \label{eq:surface_density_as_fct_of_r}
    \end{equation}
    In our case, $r_0$ labels the distance of the planet from the star and 
    $\Sigma_0:=\Sigma(r_0)$ is the surface gas density at the position of the 
    planet. The slope of $\Sigma(r)$ is characterized by the parameter $\gamma$, 
    which is assumed to be equal to 1 in this thesis.
    % which is commonly assumed to be $\approx1$, and we will do so 
    % as well in this thesis. 
    % \red{(cutoff radius?)}
%    throughout this thesis \red{(from where?)} \cite{cite}.
%    \blue{dullemond,
%    Millimeter observations probe the outer disk regions (r ?? 40 AU). They say 
%    little about the planet-forming regions (r ?? 40 AU). The surface density 
%    of matter in these inner regions is inferred by extrapolation, often 
%    simply assuming a powerlaw ?(r) ? r?3/2 or so. This powerlaw is, however, 
%    an assumption, not a fact.
%    }
    \\
    \\
    Even though the disk will be treated as a 2D object in this thesis,
    it is important to talk about the geometry of real disks observed in 
    the night sky. Contrary to what one might expect without any prerequisite 
    knowledge, observations show that the spatial extension in $z$-direction of 
    proto-planetary disks grows larger with increasing distance from the star.
    % TODO:\cite{cite}. 
    An approximation of this can be made by expressing the surface scale height
    $H$ as a polynomial function of the distance from the center $r$. With 
    $H_0:=H(r_0)$, this can be written as
    \begin{equation}
      H(r)=H_0\cdot\bigg(\frac{r}{r_0}\bigg)^{\beta}
      \label{eq:def_surface_scale_height_as_a_fct_of_r}
    \end{equation}
    Here, $\beta$ labels the so-called \textit{flaring index}. A disk with 
    $\beta>0$ is called a \textit{flared disk}. Throughout most of this thesis, 
    we will 
    assume a non-flared disk with $\Sigma(r)=\Sigma_0$, i.e. $\beta=0$.
    % \red{(TODO: observations, typical values \cite{cite})}. \\
    % TODO: \blue{flaring index $=0\Rightarrow$ constant disk width} \\
    A sketch of a non-flared as well as a flared disk can be seen in
    \autoref{fig:flaring_disk}.

    \newpage \noindent
    \autoref{eq:def_surface_scale_height_as_a_fct_of_r} suggests the definition 
    of another useful parameter, namely the \textit{aspect ratio} 
%    at a radial position $r$
    \begin{equation}
      h_r(r):=H(r)/r
    \end{equation}
    A commonly used value to set up the disk's ratio of height to width is 
    $h_r(r_0)=0.05$
    (see e.g. \citeauthor{Kley_1999}, \citeyear{Kley_1999}), 
    which will be used in most of the simulations discussed in this thesis.
%    \red{flaring index? $h_r(R=0)$, $r\in[10\ AU, 1000\ AU]$}
    \\
    \\
    % Close to the central star, volatile molecules like water and methane can not
    % exist in a stable form due to their low \red{melting points (?)}.
    % Not being able to condense into larger and heavier structures, those 
    % molecules are carried beyond the \red{\textit{frost line}} into the outer 
    % regions of the disk by the radiation pressure of the solar wind, 
    % whereas heavier elements like metals can stay closer to the center. \\
    %
    % In our model, each gas particle in the disk feels three different forces
    % acting on it. Gravitation pulls the particle inwards, inertal/centrifugal 
    % forces act in the opposite direction and forces due to gas pressure are 
    % directed outwards as well as upwards in $z$-direction. Assuming a 
    % vertical hydrostatic equilibrium, these three forces balance out.
    % \\
    % \\
    To model the disk's viscosity, the $\alpha$-description is used:
    \begin{equation}
      \nu=\alpha_{visc}\cdot\frac{c_s^2}{\Omega_K}
    \end{equation}
    It is assumed that the parameter $\alpha_{visc}$ is constant throughout 
    the disk and does not change with time. Observations of real 
    proto-planetary disk show typical values to be roughly around 
    $10^{-4}$ (e.g. \citeauthor{Dullemond_2018}, \citeyear{Dullemond_2018}), 
    whereas for numerical simulations the alpha parameter is 
    often assumed to be a few magnitudes smaller, and is commonly set to
    $\alpha_{visc}=10^{-2}$ (e.g. \citeauthor{Bailli_2016}, 
    \citeyear{Bailli_2016}). Here we remain agnostic about the different alpha 
    values and use $\alpha_{visc}=10^{-4}-10^{-2}$.
    % Reasons for this disrepancy have been described 
    % by \citeauthor{King_2007} (\citeyear{King_2007}). 
    The viscous
    evolution of the gas surface density over time is taken from
    \citeauthor{Lynden_Bell_1974} (\citeyear{Lynden_Bell_1974}), where it is 
    expressed as
    \begin{equation}
      \frac{\partial\Sigma}{\partial t}(r,t)=
      \frac{3}{r}\cdot\frac{\partial}{\partial r}
      \bigg(\sqrt{r}\cdot\frac{\partial}{\partial r}
      \nu\cdot\Sigma(r,t)\cdot\sqrt{r}\bigg)
      \label{eq:viscous_evolution}
    \end{equation} \ \\
    \\
    \\
    % tend to derive estimates for ? which are an order of magnitude smaller.
    % between $10^{-1}$ and $10^{-3}$ \cite{cite}. \\
    % \\
    % temperature of gas in disk relatively cool, between 10 and $10^3$ K.
    % \red{\SI{1e4}{\kelvin}}, assumed to be constant throughout the disk 
    % $\Rightarrow$ isothermal 
    % The total energy of a body in orbit around a central mass $M_*$ at a 
    % radial distance $r$ and a velocity vector $\vec v$ is given by the sum 
    % of the kinetic and potential energy terms. Assuming the body does not 
    % lose energy over time, 
    % \begin{equation}
    %   -G\frac{M_*}{r}+\frac{{\vec v}^2}{2}=-G\frac{M_*}{a}
    % \end{equation}
    % We saw in the observartion images in \autoref{fig:proto-planetary_disk} 
    % that planets can open up a gap in the disk. As a part of this thesis, we 
    % will investigate the influence of a planet's mass and eccentricity on the 
    % eccentricity of this gap.
    % Under the assumption of hydrostatic equilibrium, the disk's local 
    % isothermic sound velocity can be expressed as
    % \begin{equation}
    %   c_s=H\cdot\Omega_K=h_r\cdot v_{K} %=^?\sqrt{\frac{kT}{\mu m_p}}
    % \end{equation}
    % and the pressure of the gas is described by %\cite{kley_accretion_1999}
    % \begin{equation}
    %   P=c_s^2\cdot a\cdot\Sigma  
    % \end{equation}
    % This last equation will be of importance when investigating the 
    % eccentricity of the gap that forms around the planet. For this, the 
    % gap boundaries need to be determined, which can be done by calculating 
    % the maxima and minima in the gas pressure gradient.
    % % (\citeauthor{Crida_2006} (\citeyear{Crida_2006})).
    % \red{(TODO: adjust the last few sentences since I probably won't be 
    % using the pressure gradient method)}
%    \red{
%      hydrodynamical equations, 
%      model of the disk, fluid dynamics, viscosity (also numerical viscosity?) 
%    }
%    ,with the sigma slope $\gamma\approx1$ \cite{cite}
%    cutoff radius (not $r_c$, is it?) 
%    The size of proto-planetary disks depends largely on the mass of the
%    initial gas cloud \cite{cite} and \red{what are typical values?}. \\
%    \\
%    \subsection{Disk Formation}
%
%      \begin{itemize}
%        \item outcome: thin disk supported in vertical direction by gas pressure
%          % Pringle, J.E. (1981). "Accretion discs in astrophysics". Annual 
%          % Review of Astronomy and Astrophysics. 19: 137--162. 
%          % Bibcode:1981ARA&A..19..137P. doi:10.1146/annurev.aa.19.090181.001033.
%        \item can be modeled as ideal gas
%      \end{itemize}
%
%    \subsection{Star Formation in the Center of the Disk}
%
%      \begin{itemize}
%        \item accretion of gas onto star
%        \item star forming in the center of the disk
%        \item continues for about 3-10 million years
%          % Mamajek, E.E.; Meyer, M.R.; Hinz, P.M.; Hoffmann, W.F.; Cohen, M. &
    %          Hora, J.L. (2004). "Constraining the Lifetime of Circumstellar
    %          Disks in the Terrestrial Planet Zone: A Mid-Infrared Survey of
    %          the 30 Myr old Tucana-Horologium Association". The Astrophysical
    %          Journal. 612 (1): 496--510. arXiv:astro-ph/0405271. Bibcode:2004ApJ...612..496M. doi:10.1086/422550
%        \item 
%        \item
%        \item
%      \end{itemize}
%      \begin{itemize}
%        \item radii of up to 1000 AU (between 10s and 100s AU)
%        \item much more wide than thick: \blue{
%          In an accretion disc the vertical thickness $H$ is usually assumed to 
%          be small in comparison to the distance $r$ from the centre, 
%          i.e. $H/r\ll 1$. This is naturally expected when the material is in 
%          a state of near Keplerian rotation. Then one can vertically 
%          integrate the hydrodynamical equations and work only 
%          with vertically averaged state variables.}
%      \end{itemize}
    \\ 
    \\
    \\
    \\ 
    \\
    \begin{figure}[h!]
      \centering
      \begin{minipage}{.5\linewidth}
        \centering
        \subfloat[non-flared disk]{
          \includegraphics[scale=.8]{flat_disk}
        }
      \end{minipage}%
      \begin{minipage}{.5\linewidth}
        \centering
        \subfloat[flared disk]{
          \includegraphics[scale=.8]{flaring_disk}
        }
      \end{minipage}
      \caption{Qualitative sketches of the difference between flared and 
        non-flared disks
        % TODO: replace with more detailed plots
      }
      \label{fig:flaring_disk}
    \end{figure}
  
  \newpage
  \section{Planets in the Disk}

    \subsection{Planet Formation}

      Most standard planet formation theories assume the initial buildup of a 
      central rocky core \cite{Kley_1999}.
%    \red{(few Earth masses)}
      From this point on, the proto-planet may either develop into a terrestrial 
      planet or into a giant planet consisting of a rocky core and a large 
      gaseous hull. In our own Solar System, the four terrestrial planets are 
      those situated closest to the Sun, while the four gas and ice giants sit 
      in the outer regions of the Solar System. \\
      \\
      This is probably no coincidence.
      Most of the matter in the universe, and therefore most likely also most 
      matter in 
      newly formed disks, consists of light elements like hydrogen and helium.
%    disk consists either of light elements like hydrogen and helium in 
%    gaseous form or, beyond the frost line, of molecules like methane and 
%    water,
      Because of the radiation pressure of the solar wind, these light elements 
      accumulate in the outer regions of the disk. Therefore, an 
      accreting planetoid in the outskirts of the disk has much more material 
      available for accretion in its vicinity than one positioned close to the 
      center.
%      \red{(ice/frost line: which elements outside, which inside, why?)}
%    it makes 
%    sense that the inner terrestrial planets in our Solar System are much 
%    smaller than the outer gaseous giants.
      \\
      \\
%    The latter \red{happens} if the proto-planet forms outside of the
%    ice line
%    , \red{where the lighter atoms are, e.g. \ch{H2}...} 
%    \red{[cite]} \\
    %\\
%    Several stages can be distinguished during the initial planet formation 
%    process. In the beginning, the effect of gravitational interactions can 
%    be neglected, since the proto-planet forms from the material in the disk, 
%    which consists mostly of very light particles.
      In the early stages of planet formation, gravity does not play a dominant 
      role. Gas and dust grains collide and stick together mainly via 
      microphysical processes like van der Waals or Coulomb forces
%      electromagnetic \red{(Coulomb?)} forces
      (\citeauthor{Ward_1996}, \citeyear{Ward_1996}).
      These processes are still not entirely understood today.
      It is unclear how the early planet cores manage to
      grow without fragmenting into smaller pieces again at collisions with 
      other, similarly-sized planetesimals
      (\citeauthor{Morbidelli_2016}, \citeyear{Morbidelli_2016}). \\
      % . This problem is known as the 
      % meter-sized barrier 
      \\
      The early development of planetesimals will not be studied in this 
      thesis. It is assumed that the initial buildup of a planet core has 
      already taken place. From this point on, planets grow either by
      direct collisions with other bodies or via gravitational capture of gas
      (\citeauthor{Morbidelli_2016}, \citeyear{Morbidelli_2016}).
%      \red{core accretion vs. gravitational instabiliy} \\
%      \\
%      \red{Instead, ... already large planet ($\sim$ Jupiter) is placed in 
%      a disk} \\
%    van der Waals forces and electromagnetic forces, forming micrometer-sized 
%    particles; during this stage, accumulation mechanisms are largely 
%    non-gravitational in nature.[18] However, planetesimal formation in 
%    the centimeter-to-meter range is not well understood, and no 
%    convincing explanation is offered as to why such grains would 
%    accumulate rather than simply rebound.[18]:341 In particular, it is 
%    still not clear how these objects grow to become 0.1-1 km 
%    (0.06-0.6 mi) sized planetesimals;[5][19] this problem is known as 
%    the "meter size barrier":[20][21]
      This leads to the eventual formation of circumplanetary accretion disks
%    which are accretion disks of gas onto the planet, 
      similar to the proto-planetary disk itself, 
      but on much smaller scales. % \red{(how many AU approximately?)}
      \\
      \\
      % To give a sense of the time scales, ...
      % \red{for $\alpha=10^{-3}$ and $H/R=0.05$ the accumulation of mass 
      % to one Jupiter mass is on the timescale of $\approx\SI{e5}{a}$} \\
      % \\
      The accreting material comes mostly from inside the planet's Hill sphere 
      (also called the Roche sphere). % TODO: or horseshow region ?
      % \cite{cite}. 
      This is the region in space where the gravitational 
      influence of the planet's is larger than that of its central star
      (\citeauthor{Hamilton_1992}, \citeyear{Hamilton_1992}). 
      If a body with mass $m$ orbits a larger 
      mass $M$ with a semi-major axis $a$ and eccentricity $e$, the 
      radius of the Hill sphere can be approximated by
      \begin{equation}
        r_H\approx a(1-e)\sqrt[3]{\frac{m}{3M}}
        % \textnormal{\ \ \red{(why only approximately?)}}
        \label{eq:def_hill_radius}
      \end{equation}
      To simulate a proto-planetary disk, this thesis makes use of the 
      \textit{FARGO2D1D} algorithm, which will be discussed in detail in 
      \autoref{sec:fargo_algorithm}. The code handles accretion by 
      including both a Kley and a Machida accretion subroutine, 
      which are also given a detailed description in that section. Since the 
      accretion process is a highly complex process including a very large 
      number of individual particles, this can not be simulated in detail and 
      simplifications have to be made. The Kley and Machida accretion 
      constitute one way of doing this, but many different models of 
      accretion have been suggested, e.g. by \citeauthor{Matsumura_2017} 
      (\citeyear{Matsumura_2017}), \citeauthor{aless2018accretion}
      (\citeyear{aless2018accretion}) or \citeauthor{Schulik_2019} 
      (\citeyear{Schulik_2019}). \\ 
      \\
      For most simulations, the planet will be kept on a fixed orbit so that 
      there is no migration.
    
    \newpage
    \subsection{Gap Formation}
      Recent observations of proto-planetary disks often show dark, circular 
      regions around the star, as could be seen on the \textit{ALMA} images 
      in \autoref{fig:proto-planetary_disk}.
      % \cite{ALMA_HL_Tau}\cite{ALMA_TW_Hydrae}.
      It is assumed that these are created by proto-planets that form in orbit 
      around the star. When sufficiently grown, a 
      proto-planet exerts tidal torques on the disk and thereby induces 
      trailing spiral shocks (\citeauthor{Kley_1999}, 
      \citeyear{Kley_1999}). Via these, angular momentum is 
      transfered between the disk and the planet, which has the effect of 
      material being pushed away from the proto-planet. To be more precise, 
      the planet loses some of its angular momentum 
      to the outer part of the disk, while it receives some from the inner 
      part. \\
      \\
      This process 
      eventually leads to the the opening of a gap in the disk. The criteria
      for gap opening depend on the the viscosity, pressure 
      and planetary mass. They are discussed in detail in 
      Lin \& Papaloizou (1986, 1993) as well as \citeauthor{Crida_2006} 
      (\citeyear{Crida_2006}). The width of the gap, i.e. 
      its extent in radial direction, also depends on the mass of the 
      planet, as well as the viscosity and gas pressure in the disk 
      (\citeauthor{Lin_Papaloizou_1993}, \citeyear{Lin_Papaloizou_1993}). \\
      \\
      After the formation of the gap, there is still accretion from regions 
      outside or inside the planet's orbital radius, since the internal 
      evolution of the disk tends to spread the gas back into the void regions
      by diffusion (\citeauthor{Kley_1999}, 
      \citeyear{Kley_1999}). Thus, the accretion rate after gap 
      opening has occurred is further influenced by the thickness of the disk 
      as well as the viscosity of the gas.
      % \begin{itemize}
              % (..., Lin \& Papaloizou 1980, Goldreich \& Tremaine 1980, 
              % and Papaloizou \& Lin 1984). 
        % \item These transport angular momentum and material is pushed 
        %       away from the proto- planet, a process which leads eventually to 
        %       the opening of a gap in the disk (cite). The detailed criteria of 
        %       gap opening are given by 
        %       \red{also Crida 2006?}
        % \item continued accretion through the gap \cite{cite}:
        %       even after the formation of the gap, there is still accretion
        %       from regions outside or inside the planet's orbital radius
        %       $\Rightarrow$ rate depends on the viscosity of the gas in the disk
        %       as well as the disk's thickness
        % \item \blue{The modification of the disk density is the result of the 
        %   competition of torques exerted on the disk by the planet and 
        %   by the disk itself. \cite{crida}}
      % \end{itemize}
%      \blue{Du koenntest da noch sagen, dass die viskositaet die Gap
%      schliessen will und dies dann auch dazu fuehrt, dass mehr gas in
%      die Horseshoe region fliessen kann, was dafuer sorgt,
%      dass der Planet mehr akkretieren kann.}

%    \subsection{Terrestrial Planets \& Gas Giants} 
%      \begin{itemize}
%        \item in our solar system: inner rocky planets, outer gaseous planets
%        \item probably no coincidence
%        \item inner terrestrial planets, outer gaseous planets (gaseous planets 
%          beyond the frost line much larger, since there is much more ice 
%          than heavier elements $\Rightarrow$ could grow large enough to 
%          capture abundant \ch{H2} and \ch{He})
%      \end{itemize}
%
%      \myparagraph{Formation of Terrestrial Planets}
%
%        \begin{itemize}
%          \item inner regions
%          \item heavier elements
%          \item inside the frost line
%          \item
%          \item
%        \end{itemize}
%
%      \myparagraph{Formation of Gas Giants}
%
%        \begin{itemize}
%          \item 
%          \item 
%          \item 
%        \end{itemize}

%    \newpage 
%\section{Definitions}
%
%  \myparagraph{Standard Gravitational Parameter}
%  The \textit{standard gravitational parameter} $\mu$ is defined as 
%  \begin{equation}
%    \mu:=G\cdot M
%  \end{equation}
%  Here, $G$ is the Newtonian constant of gravitation and $M$ the mass of the
%  disk's central star. 
%
%\section{Laws of Planetary Motion}
%
%  \myparagraph{The \textit{vis viva} equation}
%  To model the trajectories of orbiting bodies, the following relation can be
%  used\cite{vis_viva_equation}:
%  \begin{equation}
%    v^2=GM\bigg(\frac{2}{r}-\frac{1}{a}\bigg)
%  \end{equation}
%  For a circular orbit, the approximation $r\approx a$ holds. With
%  $\Omega=\frac{v}{r}$ this leads to
%  \begin{equation}
%    \Omega^2=\frac{GM_\odot}{a^3}
%  \end{equation}
%  which can also be easily derived by equating the gravitational and 
%  centripetal forces.
%  If, on the other hand, the eccentricity $e\neq0$, then
%  \begin{equation}
%    \Omega^2=\frac{GM}{r^2}\bigg(\frac{2}{r}-\frac{1}{a}\bigg)
%  \end{equation}
%
%\section{Fluid Dynamics}
%
%\section{proto-planetary Disks}
%
%  \subsection{General Model/Approximations \red{(which headline?)}}
%    \begin{itemize}
%      \item observations
%      \item mathematical description
%        \begin{itemize}
%          \item fluid dynamics
%          \item what kind of approximations are necessary?
%        \end{itemize}
%      \item numerical approach
%    \end{itemize}
%
%  \newpage
%    \begin{itemize}
%      \item gas is supported by gas pressure $\Rightarrow$ orbits slightly 
%        more slowly as it would if it were purely Keplerian
%    \end{itemize}

    \subsection{Migration}
      The interaction of a growing proto-planet with the gas in the disk 
      can lead to a change in the orbital parameters of the proto-planet, 
      most importantly a change in the planet's semimajor axis, i.e. 
      migration of the planet (\citeauthor{Kley_2015}, \citeyear{Kley_2015}).
      Depending on the mass of the planet, different categories can be 
      distinguished to describe the migration. \\
      \\
      For low-mass planets not massive enough to open up a gap in the disk 
      (less than about 50 Earth masses), the occuring change in the 
      orbital elements is commonly labeled as \textit{type I migration}.
      % The total torque acting on a planet is given 
      % by the sum of Lindblad torques and the corotation torques generated by 
      % the gas flow in the coorbital horseshoe region \cite{Kley_2015}
      % \cite{Paardekooper_2010}. 
      We will not focus on this type of migration,
      since all planets in this thesis have masses on similar orders of 
      magnitude as Jupiter. \\
      \\
      For massive planets that are able to form a gap, 
      \textit{type II migration} occurs. 
      During the migration period, the gap will move through the disk with the 
      planet. It is often assumed that in an equilibrium situation, the gap 
      moves at exactly the same speed as the planet, and that the planet is 
      locked in the middle of the gap to maintain torque equilibrium 
      (\citeauthor{Kley_2015}, \citeyear{Kley_2015} and 
      \citeauthor{Papaloizou_1986}, \citeyear{Papaloizou_1986}) and thus 
      migrates with the 
      viscous evolution of the disk. However recent simulations done by
      e.g. \citeauthor{Kley_2015} (\citeyear{Kley_2015}) or 
      \citeauthor{Robert_2018} (\citeyear{Robert_2018})
      have shown that this is not true. In particular the studies by
      \citeauthor{Kanagawa_2018} (\citeyear{Kanagawa_2018}) 
      indicate that the migration rate scales with the depth of the gap.

