
\chapter{Introduction}

  \begin{itemize}
    \item one of the oldest questions: where do we come from? what is our 
      relevance in the cosmos? is there a connection between our home 
      (home planet) and the vastness of the night sky? are we alone?
      (maybe quote from/reference to Carl Sagan)
    \item ancient Babylonians identified the planets (meaning wanderer)
      as different from the 'fixed' background stars
    \item now, we know this is due to the planets' inependent motion around 
      their parent star, our Sun, but:
    \item for a long time $\Rightarrow$ geocentric model of the 
      universe (e.g. Ptolemy)
    \item Aristarchus of Samos had suggested it as early as 250 BC, but the 
      theory was not widely accepted until the 17th century
    \item Renaissance: heliocentric model (Copernicus), planets travel on 
      concentric circles around their parent star
    \item observations of Mars orbit by Johannes Kepler and his mentor 
      Tycho Brahe eventually lead to a mathematical description of 
      the astronomical motions
    \item improved model relative to Copernicus, ellipses instead of circles
    \item Kepler's discoveries lay foundation for Newton's work later on
    \item Kepler problem leads to the development of calculus by 
      Isacc Newton and Gottfried Wilhelm Leibniz
    \item Newtonian theory of gravity (same laws govern here and there)
    \item from the heliocentric world view, questions arise: \\
      are all stars suns like our own? \\
      are there other planets out there?
    \item more planets inside our own solar system were discovered after the 
      invention of (semi-)modern optical telescopes
    \item first observations of extrasolar planets (exoplanets) in the 
      late 20th century
    \item many have been discovered since then, most notably by the Kepler 
      space telescope, (about 2600 planets detected by KST alone, ~ 5000 total)
    \item observations show that about 1 out of every 5 Sun-like stars has an 
      Earth-sized planet inside the Goldilocks zone 
      (where liquid water can exist in a stable form) \\
      $\Rightarrow$ there could be many dozens of billions of Earth-like planets 
      in the Milky Way alone
    \item masses of observed planets range between about twice the Moon and 
      about 30 times Jupiter (multiple orders of magnitude) \\
      $\Rightarrow$ very different material compositions
      (terrestrial vs. gas giant)
    \item not only exoplanets observed, but also newly forming solar systems in 
      various stages of development (protoplanetary disks, show images)
    \item since the early 1980s studies of young stars have shown them to be 
      surrounded by cool discs of dust and gas, as so-called 
      nebular hypothesis states
    \item first image of Protoplanetary disk (PDS 70b) was reported in July 2018
    \item gives us insights into the evolution of extra-solar systems and the
      planets within $\Rightarrow$ information about our own home 
      planet/solar system (, its significance in the grander scheme of things)
      and ultimately, deep philosophical questions like whether there is 
      extraterrestrial life somewhere out there
    \item formation of solar systems/planets not entirely understood yet
      $\Rightarrow$ active field of study, complex topic 
      (gravity, magneto-hydro-dynamics, material science (?), ...)
    \item temporal limits, this bachelor's thesis can only cover a tiny portion 
      of what there is to be said about this complex and actively evolving field 
      of study
    \item this work utilizes construct a highly simplified model of a 
      protoplanetary disk (2D) with the help of computer simulations. 
      it is assumed that a ... planet has already formed 
      (circumventing problem of early accretion, cite) that is orbiting a central 
      proto star
    \item the evolution of the planet is obviously greatly 
      influenced/characterized by the rate of mass accretion. 
    \item variation of disk and planet parameters $\Rightarrow$
      influence on evolution of planet
  \end{itemize}
  
