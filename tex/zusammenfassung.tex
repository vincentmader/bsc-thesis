
\section*{Zusammenfassung}
  Bei der Modellierung von protoplanetaren Scheiben werden häufig numerische 
  Algorithmen verwendet, um die zur Entstehung von Planeten beitragenden 
  Prozesse zu simulieren. Hierzu ist es notwendig, eine Akkretionsroutine
  zu formulieren, die die Ansammlung von Gas auf Planeten in der Scheibe 
  beschreibt. Herkömmlicherweise wird die Akkretionsrate meist unabhängig von 
  der Exzentrizität des Planetenorbits berechnet
  (e.g. \citeauthor{Ida_2018} \citeyear{Ida_2018}, \citeauthor{Benz_2014} 
  \citeyear{Benz_2014} and \citeauthor{Mordasini_2012} 
  \citeyear{Mordasini_2012}).
  . \\
  \\
  In dieser Bachelorarbeit wird ein vereinfachtes Modell einer protoplanetaren 
  Scheibe mithilfe des \textit{FARGO2D1D}-Codes aufgebaut, mit dem Studien zu 
  verschiedenen Parametern von Scheibe und Planeten durchgeführt werden. 
  Ziel ist es zu zeigen, 
  dass die Exzentrizität des Orbits eine große Rolle für die 
  Gas-Akkretionsrate auf einen Planeten spielen kann. Dies liegt daran, dass 
  exzentrische Planeten durch den Austausch von Drehimpuls mit der Scheibe 
  eine deutlich breitere Lücke erzeugen, als es bei einem zirkulären 
  Orbit der Fall wäre. Diese Lücke in der Scheibe ist dementprechend 
  weniger tief, was zur Folge hat, dass die Gasdichte innerhalb der Hill-Sphäre 
  des Planeten deutlich höher ist. Der Planet hat somit mehr Material in seiner
  direkten Umgebung und die Akkretionsrate steigt an. \\
  \\
  Planeten auf elliptischen Orbits erhalten ihre Exzentrizität meist 
  auf eine von zwei Arten: entweder, wenn der Planet massiv genug ist,
  durch Interaktion mit der Scheibe oder durch gravitative Wechselwirkung
  zwischen mehreren Planeten während der Gas-Phase. In beiden Fällen kann ein 
  solcher Planet deutlich schneller akkretieren als ein Planet auf einem 
  zirkulären Orbit. Zukünftige Studien sollten dies berücksichtigen,
  insbesondere bei N-body-Simulationen, bei denen die Exzentrizität schon 
  während der Gas-Phase auftreten kann.
  % Eccentric orbits of giant planets can be generated by two ways: either by 
  % massive planets that interact with their disk or by gravitational 
  % interactions between several planets during the gas phase. This 
  % study shows that if either is the case, planets on eccentric orbits can 
  % accrete gas faster than planets on circular orbits due to the less depleted 
  % planetary gap. Future simulations, especially N-body simualtions where 
  % eccentricities can already arise during the gas phase, need to take these 
  % effects into account.
  %
  % Um ein möglichst genaues Modell für die Akkretionsprozesse innerhalb 
  % protoplanetarer Scheiben zu formulieren, könnte es deshalb von Vorteil sein, 
  % diese Abhängigkeit mit in die Akkretionsroutine aufzunehmen. Dies gilt vor 
  % Allem dann, wenn Systeme um 
  % junge Sterne betrachtet werden, in denen die durchschnittliche Exzentrizität 
  % noch deutlich höher ist als in unserem Sonnensystem.
  % \red{reformulieren}
