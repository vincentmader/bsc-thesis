\chapter{Results}
  In this thesis, several studies were done regarding the accretion of gas 
  in proto-planetary disks onto planets on eccentric orbits. The 
  \textit{FARGO2D1D} algorithm was used to simulate a thin,
  locally isothermal alpha disk.
  To get familiar with the processes underlying the behavior of the simulated 
  protoplanetary disk, several parameter studies of the gap profile as well as 
  the accretion rate were made, including the gas viscosity parameter,
  flaring index of the gas density, disk aspect ratio and planet mass. 
  The observations of earlier investigations done by e.g. 
  \citeauthor{Crida_2006} (\citeyear{Crida_2006}) and 
  \citeauthor{Kley_2006} (\citeyear{Kley_2006}) could be confirmed.
  Then, the effect of a planet's orbital eccentricity on the surface gas 
  density profile as well as on the accretion rate was investigated. \\
  \\
  It could be shown that in the utilized model, a planet's orbital eccentricity 
  greatly influences the 
  gas surface density profile of the gap formed due to exchange processes of 
  angular momentum between disk and planet. Orbits of high eccentricities 
  lead to the formation of much more shallow gaps with accordingly greater 
  radial extension. \\
  \\
  This in turn directly influences the amount of gas present in the vicinity
  of the planet's orbit. To be more precise, the total amount of gas inside 
  the Hill sphere of a planet on an eccentric orbit is larger 
  % TODO: by how much?
  than it would be for a planet on a circular orbit. 
  Thus, eccentric planets experience a significantly higher rate of gas 
  accretion. \\
  \\
  Studies were also made regarding the eccentricity of the gap as a 
  function of the planet mass and orbital eccentricity. It could be observed 
  that both of these values positively influence the eccentricity of the gap.
  Reasons for this are given in \citeauthor{Hosseinbor_2007}
  (\citeyear{Hosseinbor_2007}).
  \\
  \\
  Additionally, the processes of migration and eccentricity damping were 
  investigated. In our model of the disk, the planet experiences a rapid phase 
  of type II inward migration. The rate of change of the semimajor axis 
  depends on the mass as well as the orbital eccentricity of the planet. 
  An explanation for this has been given by
  \citeauthor{Kanagawa_2018} (\citeyear{Kanagawa_2018}), who argue that deep 
  gaps lead to slower migration, since the torque exerted on the gap-opening 
  planet depends on the surface density at the bottom of the gap. A deeper 
  gap therefore leads to slower migration, which would be the case for 
  massive planets. \\
  \\
  It could be observed that eccentricity damping occurs at a much faster 
  rate for low-mass planets than 
  for ones with larger masses. This can also be explained by the depth of 
  the gap, since deeper gap with less gas lead to weaker Lindblad resonances,
  which directly affects the migration rate. 

