
\section*{Abstract}
  When trying to study the processes involved in planet formation 
  utilizing algorithms of computational fluid dynamics, it is often necessary 
  to explicitly
  specify a model for the accretion rate of gas onto a planet. 
  Various methods for this exist, but often it is assumed that the accretion 
  rate can be calculated without taking the planet's orbital eccentricity into 
  account
  (e.g. \citeauthor{Ida_2018} \citeyear{Ida_2018}, \citeauthor{Benz_2014} 
  \citeyear{Benz_2014} and \citeauthor{Mordasini_2012} \citeyear{Mordasini_2012}).
  \\
  % \\
  % \red{planet population synthesis} \\
  \\
  In this thesis, a simplified model of a proto-planetary disk is created 
  using the \textit{FARGO2D1D} algorithm, with which studies of various disk 
  and planet parameters are made. The goal of this work is to show that 
  the orbital eccentricity can play a large role in the accretion rate of disk 
  material onto a planet. This is due to the fact that planets on  
  eccentric orbits form much broader gaps in the disk via the exchange of 
  angular momentum than would be the case for a planet on a circular orbit. 
  These gaps are accordingly more shallow, leading to a higher gas density 
  in the planet's Hill sphere and therefore faster accretion. \\
  % \\
  % \red{
  % If one tries to create an accurate model of the accretion processes 
  % occurring inside proto-planetary disks, it might be advantageous to take the 
  % orbital eccentricity into account when formulating the accretion routine,
  % especially when studying systems around very young stars, where 
  % eccentricities on average are still much higher than they are in
  % our own Solar System today.
  % } \\
  \\
  Eccentric orbits of giant planets can be generated by two ways: either by 
  massive planets that interact with their disk or by gravitational 
  interactions between several planets during the gas phase. This study shows 
  that if either is the case, planets on eccentric orbits can accrete 
  gas faster than planets on circular orbits due to the less depleted planetary 
  gap. Future studies, especially N-body simulations where eccentricities 
  can already arise during the gas phase, need to take these effects into 
  account.
  % Since recent observations of extra-solar planets 
  % show them having much higher eccentricities than planet inside our own Solar 
  % System, it could be useful to incorporate 

  % In most studies of the processes involved in planet formation, the 
  % eccentricity of a proto-planet's orbit is assumed to contribute in a 
  % neglectable manner to the value of the accretion rate of gas onto the planet
  % not taken into account 

  % This bachelor's thesis focuses on 
  % - many studies on planet formation
  % - models of accretion 
  % - many don't take eccentricity into account
  % - this thesis investigate the effect of orbital eccentricity on the formation 
  %   and structure of a gap, as well as on the accretion rate
  % - in the early stages of solar systems, orbits are more eccentric than now (?)
  % - accurate models should incorporate this into their accretion subroutines


  %   \item The goal of this thesis is to investigate the interaction between 
  %     a planet's gas accretion rate and the eccentricity of its orbit.
  %   \item for differently sized planets and several different disks
